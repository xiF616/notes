\chapter{复数与复函数}
\section{复数及其运算}
第一周第一讲第一小节->第一周第二讲第一小节

数学史
复数源于一元三次方程的求解。
一元三次方程($x,y,z$是未知量):
\begin{itemize}
    \item 九章算术中有开立方操作,可以解决$x^3=a$类型的方程。
    \item 唐代王孝通在《缉古算经》(《缉古算术》)中收录了$x^3+ax^2+bx=c$的数值解法,这方程源于实际问题。
    \item 1247年,南宋秦久韶的《数学九章》给出了一般高次方程的数值解法。欧洲人Paolo Ruffini在600年后重新发现了这一方法。
    \item 11世纪波斯数学家Omar Khayyam(海亚姆)发现一元三次方程通常有3个根,并对一元三次方程进行了分类等工作,但没有导出求根公式。Khayyam在天文、几何等方面也有贡献。
    \item 印度人等也对一元三次方程有所研究。
    \item 意大利Ferro(费罗)、Tartaglia、Cardano.
    \item Bombelli仔细研究了Cardano公式,发现其中不可避免地出现复数。Bombelli创造了复数。
\end{itemize}
考虑
\begin{equation*}
    ax^3+bx^2+cx+d=0,a\neq 0.
\end{equation*}
让$x=y-b/(3a)$, 方程化为
\begin{equation*}
    ay^3+py+q=0.
\end{equation*}
注意到
\begin{align*}
    y^3+u^3+v^3-3yuv&=(y+u+v)(y^2+u^2+v^2-yu-yv-uv)
    \\ &=(y+u+v)(y+\omega u+\omega^2 v)(y+\omega^2 u+\omega v),
\end{align*}
其中$\omega=(-1+\sqrt{3}\imageunit)/2$, $\omega=(-1-\sqrt{3}\imageunit)/2$, 只需再用Vi\`eta定理解关于$u,v$的方程组
\begin{equation*}
    u^3+v^3=q, -3uv=p
\end{equation*}
即可。最终结果为
\begin{equation*}
    y_1=u+v,y_2=\omega u+\omega^2 v,y_3=\omega^2 u+\omega v,
\end{equation*}
其中
\begin{equation*}
    u,v=\sqrt[3]{-q/2\pm\sqrt{\Delta}},\Delta=(q/2)^2+(p/3)^3.
\end{equation*}
$\Delta$称为一元三次方程的判别式:
\begin{itemize}
    \item $\Delta>0$时,$y_1\in\realset$, $y_{2,3}\in\complexset\setminus\realset$.
    \item $\Delta<0$时,由$\conjugate u=v$知道$y_1\in\realset$, 由$\conjugate{\omega u}=\omega^2 v$知道$y_2\in\realset$, 同理$y_3\in\realset$.
    \item $\Delta=0$时$u=v$, 有重根。
\end{itemize}
例(Bombelli)
\begin{equation*}
    x^3-15x-4=0.
\end{equation*}
试根:从$\pm 1,\pm 2,\pm 4$中试出$4$是根,多项式除法之后得到答案$4,-2\pm\sqrt{3}$.\\
求根公式:$p=-15,q=-4$, $\Delta=-121$, $u,v=\sqrt[3]{2\pm 11\imageunit}=2\pm i$, 之后得到根$4,-2\pm\sqrt{3}$.

例. 让$y=\cos 20^\circ$, 三倍角公式导致$4y^3-3y=0.5$, 再让$z=2y$得到$z^3-3z-1=0$. $p=-3,q=-1$, $\Delta=-3/4$, $u,v=\sqrt[3]{0.5\pm\imageunit 0.5\sqrt{3}}$. 几何上看出$u,v=\cos 20^\circ\pm \imageunit\sin 20^\circ$. $z$的3个根为$z_1=2\cos 20^\circ,z_2=2\cos 140^\circ,z_3=2\cos 260^\circ$.

问题:能否只在$\realset$中运算得到上例中的三个根?\\
注意到$\cos 140^\circ=-\cos 40^\circ=-2(\cos 20^\circ)^2+1$, $z_{2,3}$都是$z_1$的多项式。取$F=\rationalset,E=\rationalset[z_1]$, $E$就是关于$z$的多项式$z^3-3z-1$的分裂域,$\operatorname{Gal} (E/F)=\integerset_3$. 可能需要修改的Galios理论,根式扩张但不能含有单位根,只能对$\geqslant 0$的数开方。不再深入。

不限于根式?

对于有三个实根的一元三次方程$y^3+py+q=0$, $\Delta=(q/2)^2+(p/3)^3<0$. 取$\lambda=\pm\sqrt{-4p/3}\in\realset$, 作尺度变换$y=\lambda z$得到$4z^3-3z=-4q/\lambda^3:=\gamma$, 有$|\gamma|<1$. 令$\gamma=\cos\theta$, 得到$z=\cos (\theta+(2k\pi)/3),k=0,1,2$.

对一元四次方程进行变形,可以归结为一元三次方程问题。

一元五次方程可以用椭圆函数求解。
\begin{equation*}
    \arcsin x=\int_{0}^{x}(1-t^2)^{-1/2}dt
\end{equation*}
椭圆积分
\begin{equation*}
    \int P(x)^{-1/2}dx,
\end{equation*}
$P$是三次或四次多项式。椭圆积分的反函数是椭圆函数,因此椭圆函数可以看作三角函数的推广。

更高次方程:引入代数函数。

\section{复平面与Riemann球}
第一周第二讲第一小节->

$\complexset$上的度量、拓扑同$\realset^2$.

设$E\subseteq\complexset$. $E$中的一条道路是指一个连续映射$\gamma\colon [0,1]\to E$. 称$E$道路连通,如果$\forall p,q\in E$, 存在$E$中的道路$\gamma$满足$\gamma(0)=p,\gamma(1)=q$.
% $\vcentcolon\Leftrightarrow$

定理:对于$\complexset$中的开集,连通和道路连通等价。

定义:连通的开集叫作区域,区域的闭包(区域并上它的边界)叫作闭区域。

\begin{theorem}
    设$D\subseteq\complexset$是区域,$p,q\in D$, 那么存在分段线性道路$\gamma\colon [0,1]\to D$使得$\gamma(0)=p,\gamma(1)=q$. $\gamma$分段线性是指,存在对区间$[0,1]$的分划$0=t_0<t_1<t_2<\dots<t_n=1$和常数$z_{i,0},z_{i,1},i=1,2,\dots,n$ 使得$\gamma(t)=z_{i,0}+z_{i,1}t,\forall t\in [t_{i-1},t_i]$.
\end{theorem}
\begin{proof}
    取连续函数$\tilde\gamma\colon [0,1]\to D$满足$\tilde\gamma(0)=p,\tilde\gamma(1)=q$, 那么$\Gamma=\tilde\gamma([0,1])$是紧集,因此是闭集。因为闭集$\complexset\setminus D$与$\Gamma$不相交,所以有$\varepsilon\coloneq d(\complexset\setminus D,\Gamma)>0$. 设$\tilde{\gamma}(t)=\tilde{x}(t)+\imageunit\tilde{y}(t)$, $\tilde{x}$, $\tilde{y}\in C([0,1],realset)$. 用一致连续性,存在$\delta>0$, 使得$t',t''\in [0,1]$时有$\tilde{x}(t')-\tilde{x}(t'')<\varepsilon$, $\tilde{y}(t')-\tilde{y}(t'')<\varepsilon$. 将$[0,1]$区间$n$等分,其中$n^{-1}<\delta$, 并在第$k$个小区间上用直线段连接$\tilde{\gamma}((k-1)/n)$和$\tilde{\gamma}(k/n)$得到$\gamma(t)=x(t)+\imageunit y(t)$. 现在对于$t\in [t_{k-1},t_k]$, 估计
    \begin{align*}
        |x(t)-\tilde x(t)|&\leqslant |x(t)-\tilde x(t_k)|+|\tilde x(t_k)-\tilde x(t)|
        \\ &\leqslant |x(t_{k-1})-\tilde x(t_k)|+\varepsilon
        \\ &=|\tilde x(t_{k-1})-\tilde x(t_k)|+\varepsilon<2\varepsilon.
    \end{align*}
    同理$|y(t)-\tilde y(t)|<2\varepsilon$. 于是$|\gamma(t)-\tilde{\gamma}(t)|<C\varepsilon$. 调整常数让$|\gamma(t)-\tilde{\gamma}(t)|<\varepsilon$, 并由$\varepsilon$的定义知道折线落在$D$中。%\qedhere
\end{proof}

定义(可求长曲线):设曲线$\gamma\colon [0,1]\to\complexset$. 对任意分划$P\colon 0=t_0<t_1<t_2<\dots<t_n=1$, 定义$L(\gamma,P)\coloneq\sum_{k=1}^{n}|\gamma(t_{k-1})-\gamma(t_k)|$. 若$L(\gamma)\coloneq\sup \{L(\gamma,P)\colon P\text{是对}[0,1]\text{的分划}\}<+\infty$, 那么称$\gamma$是可求长的,长度是$L(\gamma)$.

若$\gamma$分段光滑,那么$\gamma$可求长,且$L(\gamma)=\int_{0}^{1}|\gamma'(t)|dt=\int_{0}^{1}\sqrt{x'(t)^2+y'(t)^2}dt$.

实分析:$\gamma$可求长$\Leftrightarrow$ $x,y$有界变差。

曲线积分要求曲线可求长。后面会比较关注边界是可求长曲线的区域。

\begin{theorem}
    (Jordan) 设$\gamma\colon [0,1]\to\complexset$是简单(不自交)闭(起点和终点相同)曲线,即$\gamma(t_1)=\gamma(t_2)\Leftrightarrow \{t_1,t_2\}=\{0,1\}$, 那么$\complexset\setminus\gamma$有两个连通分支,这两个连通分支都是开集,其中一个是有界的,称为$\gamma$所围的区域的内部$\operatorname{Int}\gamma$,另一个是无界的,称为$\gamma$所围的区域的外部$\operatorname{Ext}\gamma$.
\end{theorem}